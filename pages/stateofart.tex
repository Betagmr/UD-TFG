\section{Antecedentes y justificación}
El objetivo de esta seccion es describir las condiciones del entorno en el que se lleva a cabo 
el proyecto, se recopilarán datos e información sobre la situación actual en lo que respecta al FJSP, 
incluyendo la utilización de diferentes tecnicas que han sido utilizadas historicamente para su resolución. 
Para ello, se revisará el estado del arte y las últimas tendencias en este campo, con el fin de obtener una 
mejor comprensión del problema y explicar la oportunidad de realizar nuevos aportes en esta área de investigación.


\subsection{Estado del arte}
\subsubsection{Metodos exactos}
Los métodos exactos se basan en la resolución matemática del problema, y pueden garantizar la 
obtención de soluciones óptimas en teoría. Sin embargo, estos métodos suelen ser computacionalmente 
costosos y, por lo tanto, se aplican principalmente a problemas de tamaño reducido.\medskip

En el caso del FJSP, algunos métodos exactos que se han utilizado incluyen la programación 
lineal de enteros mixta (MILP)\cite{milp}, que modela el problema como un conjunto de restricciones lineales y 
una función objetivo lineal, y la programación por restricciones (CP)\cite{wikiCP}, que reduce el problema a
un conjunto de restricciones lógicas y utiliza técnicas de búsqueda para encontrar soluciones que satisfagan 
todas las restricciones. Existen aplicaciones hechas por google como Ortools\cite{ortools}
o por IBM como CPLEX\cite{cplex}, que son ejemplos de soluciones que utilizan estos métodos exactos para resolver
problemas de optimización. Estas aplicaciones son muy útiles para resolver instancias de tamaño reducido,
pero no son capaces de resolver instancias de tamaño grande, debido a la gigantesca dimensionalidad que
implica el resolver este tipo de problemas por métodos exactos.


\subsubsection{Metodos heurísticos}
Un método heurístico es un algoritmo o técnica de búsqueda que está diseñado para encontrar 
soluciones aproximadas a problemas de optimización, cuando las soluciones exactas son difíciles 
de encontrar. Los métodos heurísticos son útiles cuando el tiempo y los recursos son limitados, 
y cuando el problema a resolver es demasiado complejo para ser resuelto por medios exactos. 
Estos métodos se basan en la experiencia y el conocimiento previo para generar soluciones que 
son buenas pero no necesariamente óptimas. Los métodos heurísticos pueden ser aplicados a una 
amplia variedad de problemas de optimización, incluyendo el FJSP.\medskip

En el caso del FJSP, existen diferentes métodos heurísticos que se han utilizado con éxito 
para encontrar soluciones aproximadas. Un ejemplo es el algoritmo de búsqueda tabú, 
que se basa en la exploración de soluciones vecinas para encontrar soluciones cada vez mejores. 
Otro ejemplo es el algoritmo de búsqueda local, que busca soluciones iterativamente en una 
vecindad de soluciones. También se ha utilizado el enfoque basado en la construcción de 
soluciones parciales, donde se construyen soluciones en etapas, utilizando información previa 
y reglas heurísticas para guiar el proceso de construcción. Todos estos métodos heurísticos 
han demostrado ser efectivos para resolver el problema de FJSP, y se pueden adaptar y mejorar 
para abordar variantes más complejas del problema.

\subsubsection{Algoritmos genéticos}
Los algoritmos genéticos son una técnica de optimización inspirada en la evolución biológica.
Estos algoritmos imitan el proceso de selección natural, reproducción y mutación en la búsqueda 
de soluciones óptimas o cercanas a la óptima para problemas de optimización combinatoria. La 
idea básica de los algoritmos genéticos es utilizar la selección natural para encontrar soluciones
mejores a través de la generación de nuevas soluciones a partir de soluciones existentes y la 
aplicación de operadores de cruce y mutación.\medskip

En la literatura, estos algoritmos han sido ampliamente utilizados y han demostrado ser 
efectivos para encontrar soluciones de alta calidad. La representación cromosómica utilizada 
en los algoritmos genéticos se basa en la codificación de una solución del problema de FJSP 
como una cadena de genes, y los operadores genéticos como el cruzo o la mutación, se aplican 
para producir nuevas soluciones o realizar ligeros cambios en aquellas ya existentes. Una de 
sus principales ventajas es que pueden manejar fácilmente múltiples objetivos y restricciones 
adicionales, y son capaces de explorar ampliamente el espacio de soluciones.\medskip

En cambio uno de sus problemas es que los algoritmos genéticos pueden requerir una gran cantidad 
de tiempo y esfuerzo para encontrar soluciones aceptables, y no siempre garantizan encontrar la 
mejor solución posible. Además, pueden requerir ajustes cuidadosos de los parámetros y una 
cuidadosa selección de operadores para funcionar bien en diferentes instancias del problema y 
tienen tendencia a sufrir de prematura convergencia lo que genera dificultades para escapar de 
óptimos locales.

\subsubsection{Algoritmos híbridos}
Los algoritmos híbridos son aquellos que combinan dos o más algoritmos que resuelven el mismo problema. 
En el caso del FJSP, los algoritmos híbridos combinan diferentes técnicas de optimización para encontrar 
soluciones. Por ejemplo, un algoritmo híbrido podría combinar un algoritmo genético con un algoritmo de 
búsqueda local para aprovechar las ventajas de ambos y mejorar la calidad de las soluciones encontradas.

\subsubsection{Redes neuronales}
Las redes neuronales son una clase de algoritmos de aprendizaje automático inspirados en la 
estructura y el funcionamiento del cerebro humano. Estas redes están formadas por múltiples 
capas de neuronas interconectadas que procesan información para resolver problemas de clasificación, 
regresión, reconocimiento de patrones, entre otros.\medskip

En el contexto del FJSP, las redes neuronales pueden ser utilizadas para aprender patrones en los datos
de entrada del problema, como la secuencia de tareas y las restricciones de precedencia. Esto puede 
ayudar a generar soluciones de alta calidad al FJSP, incluso para instancias de problemas grandes y 
complejas. Además, las redes neuronales también pueden ser entrenadas para mejorar la calidad de las 
soluciones generadas por otros algoritmos de optimización, como los algoritmos genéticos y los métodos 
heurísticos.\medskip

La principal ventaja de las redes neuronales es su capacidad para manejar datos no lineales y ruidosos, 
y para detectar patrones complejos en los datos de entrada. Sin embargo, las redes neuronales también 
pueden presentar desafíos en el entrenamiento y ajuste de los hiperparámetros, y pueden ser propensas 
a sobreajuste si no se controlan adecuadamente, pero incluso con esos inconvenientes son una de las 
técnicas más prometedoras para el FJSP y pueden ser utilizadas en combinación con otras técnicas de 
optimización para mejorar la calidad de las soluciones.

\subsubsection{Reinforcement learning}
El aprendizaje por refuerzo es una técnica de aprendizaje que se basa en la idea de que un agente 
interactúa con un entorno para aprender a tomar decisiones óptimas en función de las recompensa 
recibida por sus acciones. En el aprendizaje por refuerzo, el agente aprende a través de la 
experiencia, probando diferentes acciones y observando las recompensas asociadas con cada una 
de ellas, con el objetivo de maximizar la recompensa total a largo plazo.\medskip

En el contexto del FJSP, el aprendizaje por refuerzo puede ser utilizado para entrenar a un 
agente que toma decisiones secuenciales sobre la asignación de tareas a máquinas y la 
planificación de la secuencia de tareas, con el objetivo de minimizar el tiempo total de producción. 
Una de las principales ventajas del aprendizaje por refuerzo es su capacidad para aprender directamente 
de la experiencia, sin la necesidad de un modelo matemático explícito del problema, lo que lo hace 
útil para resolver problemas complejos y mal definidos como el FJSP. Sin embargo, el aprendizaje 
por refuerzo también presenta desafíos en la definición de la función de recompensa y en la 
selección de la política óptima que maximiza la misma. Además, requiere de una gran cantidad de tiempo 
y recursos computacionales para entrenar al agente en instancias del problema de FJSP grandes y complejas.

\subsubsection{Imitation learning}
El aprendizaje por imitación es una técnica de aprendizaje por refuerzo en la que un modelo aprende
a partir de ejemplos proporcionados por un experto. En el aprendizaje por imitación, el modelo trata
de imitar la forma en que el experto realiza una determinada tarea, utilizando los ejemplos como guía.
Además, el experto no tiene por qué ser una persona física, también es posible utilizar algoritmos de
optimización para generar soluciones de alta calidad, y luego utilizar estas soluciones como ejemplos
de entrenamiento para el aprendizaje por imitación.\medskip

Esta técnica puede ser utilizada para aprender a generar soluciones de alta calidad a partir de 
configuraciones previamente resueltas por un experto, esto es gracias a que el modelo es capaz de 
identificar patrones y características comunes en las soluciones óptimas del problema. Luego, el 
modelo puede ser utilizado para generar nuevas soluciones que imiten el comportamiento del experto 
humano. Una de las principales ventajas del aprendizaje por imitación es que los datos que se le 
ofrece son soluciones de alta calidad, lo que agiliza el entrenamiento sin requerir la optimización 
iterativa del problema. 

\subsubsection{Ensenble learning}
Un ensamblador en machine learning, también conocido como ensemble learning, es una técnica que 
combina múltiples modelos de aprendizaje automático para mejorar la precisión y estabilidad de 
las predicciones. Cada uno de estos algoritmos puede tener fortalezas y debilidades en términos 
de su capacidad para encontrar soluciones óptimas en diferentes instancias del problema. Al 
combinarlos en un ensamblador, se pueden aprovechar las fortalezas de cada uno de ellos y obtener
soluciones más robustas y de mejor calidad.\medskip

Cada modelo produce una predicción diferente y las predicciones de los distintos modelos se combinan 
para obtener una única predicción. Existen diferentes tipos de ensemblers como 
el ensamble por votación, el ensamble por bagging, el ensamble por boosting y el ensamble por stacking.
Cada uno de estos ensemblers representa una estrategia diferente para combinar las predicciones de
los distintos modelos, por ejemplo en el ensamble por votación, cada modelo vota por la accion a la que predice, 
y la resultante con mayor número de votos es la predicción final.

\subsection{Antecedentes}
Una vez terminada la investigación sobre el estado actual del problema, se explorarán diversas ideas 
y reflexiones que han supuesto un impacto previo para la motivación de este trabajo. A nivel de industria, se 
presentan dos principales desafíos que enfrenta este problema de optimización combinatoria: la utilidad de los 
algoritmos metaheurísticos en escenarios online, y la capacidad de los usuarios para expresar claramente las 
funcionalidades y restricciones del problema. En ambos casos, se explicaran los antecedentes que han motivado
el desarrollo de este trabajo y se presentarán las principales ideas que se han explorado en la literatura.\medskip

De acuerdo a la investigacion realizada con anterioridad, si bien los metaheurísticos son ampliamente utilizados 
en la solución de problemas de optimización, su velocidad y practicidad en situaciones reales pueden ser limitadas, 
ya que a menudo requieren de una gran cantidad de iteraciones y evaluaciones para encontrar una solución óptima. 
Esto puede resultar en un tiempo de ejecución prolongado, lo que no es adecuado para aplicaciones en tiempo real 
o decisiones a corto plazo. El buscar un sistema que pueda resolver el problema en tiempo real es un desafío 
importante para la industria, ya que la capacidad de tomar decisiones rápidas y eficientes es crucial.\medskip 

Otro de los desafíos que enfrenta la industria es la capacidad de los usuarios para expresar claramente las
funcionalidades y restricciones del problema. En la mayoría de las situaciones, los usuarios no tienen conocimientos
suficientes o la capacidad de expresar las restricciones del problema de forma precisa, pero si son capaces de
indentificar las soluciones óptimas. Por ejemplo, en el caso de la logística en el transporte, donde una empresa 
de transporte de mercancías puede tener una idea clara de cómo quisiera asignar los vehículos a las rutas de 
entrega para minimizar los tiempos de viaje y los costos de combustible. Sin embargo, expresar todas las 
restricciones y consideraciones relacionadas con la capacidad de los vehículos, los horarios de los conductores, 
las restricciones de tiempo y las regulaciones de tráfico en una formulación matemática puede resultar complicado 
y requerir un conocimiento técnico multidisciplinario.\medskip

En este escenario, los operadores con experiencia podrían proporcionar soluciones de alta calidad, que luego 
podrían ser utilizados como datos de entrenamiento para un modelo de Imitation learning. El modelo aprendería 
a imitar el comportamiento del experto a partir de estos ejemplos, capturando así la intuición y el conocimiento 
tácito que tienen los usuarios, pero que pueden tener dificultades para expresar de manera formal. De esta manera, 
se podra abordar la brecha entre el conocimiento tácito de los operadores y la representación formal de las 
funcionalidades y restricciones del problema, permitiendo desarrollar soluciones más precisas y efectivas.\medskip

En conclusión, la idea de aprovechar el conocimiento tácito de los expertos en la resolución de problemas de 
optimización combinatoria a través de técnicas de inteligencia artificial ofrece un prometedor potencial. La 
capacidad de capturar el conocimiento de los usuarios que enfrentan dificultades en la formulación
del problema, puede ser clave para mejorar la efectividad en la toma de decisiones para procesos complejos. Además,
si gracias a estas técnicas logramos inferir una estrategia para construir soluciones a partir de ese conocimiento,
podríamos solucionar colateralmente el problema del tiempo de ejecución que habíamos identificado en los algoritmos 
metaheurísticos, ya que no necesitaríamos explorar todo el espacio de soluiciones para encontrar una solución óptima,
lo que nos permitiría reducir el tiempo de ejecución de los algoritmos drasticamente.



\pagebreak