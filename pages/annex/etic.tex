\section{Anexo 2, Consideraciones éticas}
El FJSP es un problema de optimización combinatoria en el que se deben ordenar un 
conjunto de trabajos que se distribuyen a través de un grupo de máquinas. La 
complejidad del problema radica en la variabilidad del mismo, ya que cada trabajo 
puede ser procesado en diferentes máquinas. El objetivo es minimizar el tiempo 
total de producción utilizando inteligencia artificial, de tal manera que se 
aprovechen al máximos los recursos acortando a su vez los tiempos de entrega. A 
continuación, se van a identificar diferentes cuestiones éticas de esta línea de 
investigación y las consecuencias que pueden acarrear.

\subsection{Impacto medioambiental}
Uno de los principales perjudicados de la implementación de sistemas de cálculo 
para resolver problemas de combinatoria es sin duda nuestro planeta. Generalmente 
se requieren grandes cantidades de energía y equipamiento para funcionar, 
especialmente si se están ejecutando en servidores a gran escala. El objetivo de 
mi PFG es utilizar redes neuronales de grafos, las cuales únicamente requieren de 
un desembolso energético inicial, ya que una vez terminado su entrenamiento se 
vuelven mucho menos costosas energéticamente hablando que los sistemas más populares 
instalados en las empresas. Gracias a esto, estaríamos reduciendo significativamente 
la huella de carbono que generan estos sistemas, dado que pasaríamos de consumir 
grandes cantidades de energía, debido al masivo tiempo de cómputo, a una cantidad 
prácticamente despreciable.

\subsection{Implicaciones en el empleo}
Un proyecto de este tipo podría tener fuertes implicaciones en el empleo, especialmente 
si se automatizan tareas que antes eran realizadas por trabajadores humanos. Es 
importante tener en cuenta cómo se verán afectados los trabajadores y si se tomarán 
medidas para garantizar la protección de sus derechos laborales.\medskip

Por una parte tenemos el caso principal que se da cuando una empresa destina a un 
trabajador a realizar la asignación de tareas de una forma manual, este procedimiento 
puede ser extremadamente largo llegando a registrarse tiempos de hasta 3 jornadas 
laborales para diseñar una planificación óptima. Gracias a esta tecnología podremos 
proporcionar una planificación óptima en cuestión de segundos y liberar de toda esa 
carga de trabajo que suponen este tipo de tareas.\medskip

Otro caso directamente relacionado con los trabajadores, es si una empresa intentase 
asignar estas planificaciones a personas físicas en lugar de máquinas industriales. 
Aunque es posible realizar dicha asignación, no es algo que se debiese permitir ya 
que este tipo de algoritmos están construidos desde el punto de vista de que las 
tareas las realizan máquinas, y no tienen en consideración aspectos como el estado 
mental de los trabajadores, el estrés que supone realizar una tarea, tiempos de 
descansos, etc. Es por ello que es nuestra obligación conocer de primera mano las 
condiciones en las que se piensa implementar para evitar casos de explotación laboral.


\subsection{Efecto en la calidad del producto}
La optimización de la producción puede ser beneficiosa para la empresa, pero es 
importante considerar cómo afectará a la calidad del mismo. Por mucho que se 
agilicen los tiempos de producción, si se descuida el producto final por el mero 
hecho de disminuir los tiempos de entrega nos veríamos ante la situación en la 
que los clientes resignasen de seguir consumiendo nuestro producto y, en última 
instancia, perjudicando la reputación de la empresa. Además, es importante encontrar 
un equilibrio que nos permita ofrecer productos de alta calidad de manera eficiente. 
Esto se vuelve especialmente importante si consideramos las posibles consecuencias 
que puede acarrear una mala calidad en productos relacionados directamente con la 
salud, seguridad, transporte o alimentación.

\subsection{Privacidad y seguridad de los datos}
La recopilación, almacenamiento y uso de datos personales para el entrenamiento de 
modelos de IA pueden representar un riesgo para la privacidad de los individuos y 
empresas. Por lo tanto, es necesario establecer medidas de seguridad para garantizar 
que los datos sean utilizados de manera responsable y buscar alternativas que minimicen 
la cantidad de información que se necesite extraer.\medskip

Una de las soluciones que propongo es la creación de un programa que genere de forma 
aleatoria instancias que no tengan ninguna correlación directa con los datos de las 
empresas. Generar instancias aleatorias del problema implica la creación de conjuntos 
de entrenamiento que se asemeje al problema original, pero que no contengan datos 
específicos de un individuo o empresa. Esto puede minimizar completamente su uso, ya 
que se utilizan datos generados aleatoriamente en lugar de datos reales.\medskip

Por otro lado, para los datos que se extraigan de las empresas se deben establecer 
políticas claras de privacidad y utilizar medidas de seguridad adecuadas, como 
la encriptación de datos, el acceso restringido y la gestión adecuada de las 
credenciales de acceso. Además, es importante que se cumplan las leyes y 
regulaciones aplicables en materia de privacidad y seguridad de los datos, como 
el reglamento general de protección de datos de la Unión Europea.\medskip

\subsection{Responsabilidad corporativa}
La responsabilidad corporativa en el uso de sistemas de inteligencia artificial 
es un tema cada vez más relevante en el ámbito empresarial. Si una empresa utiliza 
IA para gestionar el orden de fabricación y surge algún problema, la responsabilidad 
de ese problema dependerá de varios factores, quién diseñó el modelo, sí se toman 
decisiones autónomas o se proporcionan recomendaciones y si existen medidas adecuadas 
para mitigar el riesgo de problemas.\medskip

En nuestro caso, se muestran configuraciones a modo de recomendación y es responsabilidad 
del operario tomar la decisión final. Aun así, aunque la configuración final es validada 
por el usuario,  tenemos la responsabilidad ética de programar correctamente su funcionalidad 
para evitar que genere combinaciones incorrectas que puedan poner en riesgo a terceros. 
No solo eso, también es indispensable diseñar medidas en el caso de que se intente cambiar 
la configuración por alguna no válida, reduciendo así el posible error humano. Esto implica 
realizar pruebas exhaustivas, implementar salvaguardas adecuadas y tomar acciones para 
mitigar el riesgo de problemas. 

\subsection{Innovación científica}
Los problemas como el FJSP pertenecen a una familia de problemas matemáticos llamados NP-hard, 
estos son considerados los más difíciles de resolver en términos de tiempo y no se conoce un 
algoritmo eficiente que de un resultado exacto en tiempo polinómico. El proporcionar una nueva 
solución a un problema NP-hard es esencial para resolver otros problemas de la misma familia, 
ya que comparten estructura y propiedades similares.\medskip

Estos avances ayudan significativamente a los investigadores y a la propia comunidad científica 
que identifica este tipo de problemas dentro de la categoría de “Problemas del milenio”, los 
cuales son premiados con 1 millón de dólares. El progreso dentro de este campo es esencial para 
el avance de la ciencia y la ingeniería, ya que repercute directamente en multitud de campos tan 
variados como el transporte, la salud, la física, etc.
