\section{Anexo 2, Consideraciones éticas}
El FJSP es un problema de optimización combinatoria en el que se deben ordenar un 
conjunto de trabajos que se distribuyen a través de un grupo de máquinas. La 
complejidad del problema radica en la variabilidad del mismo, ya que cada trabajo 
puede ser procesado en diferentes máquinas. El objetivo es minimizar el tiempo 
total de producción utilizando inteligencia artificial, de tal manera que se 
aprovechen al máximos los recursos acortando a su vez los tiempos de entrega. A 
continuación, se van a identificar diferentes cuestiones éticas de esta línea de 
investigación y las consecuencias que pueden acarrear, realizar un análisis desde
diferentes corrientes éticas y proponer soluciones para mitigar los posibles
problemas que puedan surgir. 

\subsection{Cuestiones éticas}
\subsubsection{Impacto medioambiental}
Uno de los principales perjudicados de la implementación de sistemas de cálculo 
para resolver problemas de combinatoria es sin duda nuestro planeta. Generalmente 
se requieren grandes cantidades de energía y equipamiento para funcionar, 
especialmente si se están ejecutando en servidores a gran escala. El objetivo de 
mi TFG es utilizar redes neuronales de grafos, las cuales únicamente requieren de 
un desembolso energético inicial, ya que una vez terminado su entrenamiento se 
vuelven mucho menos costosas energeticamente hablando que los sistemas más populares 
instalados en las empresas. Gracias a esto, estaríamos reduciendo significativamente 
la huella de carbono que generan estos sistemas, dado que pasaríamos de consumir 
grandes cantidades de energía, debido al masivo tiempo de cómputo, a una cantidad 
prácticamente despreciable.

\subsubsection{Implicaciones en el empleo}
Un proyecto de este tipo podría tener fuertes implicaciones en el empleo, especialmente 
si se automatizan tareas que antes eran realizadas por trabajadores humanos. Es 
importante tener en cuenta cómo se verán afectados los trabajadores y si se tomarán 
medidas para garantizar la protección de sus derechos laborales.\medskip

Por una parte tenemos el caso principal que se da cuando una empresa destina a un 
trabajador a realizar la asignación de tareas de una forma manual, este procedimiento 
puede ser extremadamente largo llegando a registrarse tiempos de hasta 3 jornadas 
laborales para diseñar una planificación óptima. Gracias a esta tecnología podremos 
proporcionar una planificación óptima en cuestión de segundos y liberar de toda esa 
carga de trabajo que suponen este tipo de tareas.\medskip

Otro caso directamente relacionado con los trabajadores, es si una empresa intentase 
asignar estas planificaciones a personas físicas en lugar de máquinas industriales. 
Aunque es posible realizar dicha asignación, no es algo que se debiese permitir ya 
que este tipo de algoritmos están construidos desde el punto de vista de que las 
tareas las realizan máquinas, y no tienen en consideración aspectos como el estado 
mental de los trabajadores, el estrés que supone realizar una tarea, tiempos de 
descansos, etc. Es por ello que es nuestra obligación conocer de primera mano las 
condiciones en las que se piensa implementar para evitar casos de explotación laboral.


\subsubsection{Efecto en la calidad del producto}
La optimización de la producción puede ser beneficiosa para la empresa, pero es 
importante considerar cómo afectará a la calidad del mismo. Por mucho que se 
agilicen los tiempos de producción, si se descuida el producto final por el mero 
hecho de disminuir los tiempos de entrega nos veríamos ante la situación en la 
que los clientes resignasen de seguir consumiendo nuestro producto y, en última 
instancia, perjudicando la reputación de la empresa. Además, es importante encontrar 
un equilibrio que nos permita ofrecer productos de alta calidad de manera eficiente. 
Esto se vuelve especialmente importante si consideramos las posibles consecuencias 
que puede acarrear una mala calidad en productos relacionados directamente con la 
salud, seguridad, transporte o alimentación.

\subsubsection{Privacidad y seguridad de los datos}
La recopilación, almacenamiento y uso de datos personales para el entrenamiento de 
modelos de IA pueden representar un riesgo para la privacidad de los individuos y 
empresas. Por lo tanto, es necesario establecer medidas de seguridad para garantizar 
que los datos sean utilizados de manera responsable y buscar alternativas que minimicen 
la cantidad de información que se necesite extraer.\medskip

Una de las soluciones que propongo es la creación de un programa que genere de forma 
aleatoria instancias que no tengan ninguna correlación directa con los datos de las 
empresas. Generar instancias aleatorias del problema implica la creación de conjuntos 
de entrenamiento que se asemeje al problema original, pero que no contengan datos 
específicos de un individuo o empresa. Esto puede minimizar completamente su uso, ya 
que se utilizan datos generados aleatoriamente en lugar de datos reales.\medskip

Por otro lado, para los datos que se extraigan de las empresas se deben establecer 
políticas claras de privacidad y utilizar medidas de seguridad adecuadas, como 
la encriptación de datos, el acceso restringido y la gestión adecuada de las 
credenciales de acceso. Además, es importante que se cumplan las leyes y 
regulaciones aplicables en materia de privacidad y seguridad de los datos, como 
el reglamento general de protección de datos de la Unión Europea.\medskip

\subsubsection{Responsabilidad corporativa}
La responsabilidad corporativa en el uso de sistemas de inteligencia artificial 
es un tema cada vez más relevante en el ámbito empresarial. Si una empresa utiliza 
IA para gestionar el orden de fabricación y surge algún problema, la responsabilidad 
de ese problema dependerá de varios factores, quién diseñó el modelo, sí se toman 
decisiones autónomas o se proporcionan recomendaciones y si existen medidas adecuadas 
para mitigar el riesgo de problemas.\medskip

En nuestro caso, se muestran configuraciones a modo de recomendación y es responsabilidad 
del operario tomar la decisión final. Aun así, aunque la configuración final es validada 
por el usuario,  tenemos la responsabilidad ética de programar correctamente su funcionalidad 
para evitar que genere combinaciones incorrectas que puedan poner en riesgo a terceros. 
No solo eso, también es indispensable diseñar medidas en el caso de que se intente cambiar 
la configuración por alguna no válida, reduciendo así el posible error humano. Esto implica 
realizar pruebas exhaustivas, implementar salvaguardas adecuadas y tomar acciones para 
mitigar el riesgo de problemas. 

\subsubsection{Innovación científica}
Los problemas como el FJSP pertenecen a una familia de problemas matemáticos llamados NP-hard, 
estos son considerados los más difíciles de resolver en términos de tiempo y no se conoce un 
algoritmo eficiente que de un resultado exacto en tiempo polinómico. El proporcionar una nueva 
solución a un problema NP-hard es esencial para resolver otros problemas de la misma familia, 
ya que comparten estructura y propiedades similares.\medskip

Estos avances ayudan significativamente a los investigadores y a la propia comunidad científica 
que identifica este tipo de problemas dentro de la categoría de “Problemas del milenio”, los 
cuales son premiados con 1 millón de dólares. El progreso dentro de este campo es esencial para 
el avance de la ciencia y la ingeniería, ya que repercute directamente en multitud de campos tan 
variados como el transporte, la salud, la física, etc.

\subsection{Análisis desde diferentes corrientes éticas}
\subsubsection{Atendiendo a criterios utilitaristas}
La corriente ética del utilitarismo sostiene que las acciones deben evaluarse en 
términos de su capacidad para producir el mayor bienestar o felicidad posible para 
la mayor cantidad de personas afectadas por ellas. Podemos identificar bajo este punto 
tres posibles acciones relacionadas con la aplicación del FJSP y analizarlas independientemente.

\begin{enumerate}
    \item \textbf{Emplear el modelo sin restricciones:} Esto significa que el algoritmo 
    se aplicaría sin tener en cuenta el bienestar de los trabajadores. Siguiendo esta premisa, 
    es posible que el algoritmo optimice la programación de trabajos de manera tal que los 
    trabajadores deban realizar tareas muy exigentes o en horarios desfavorables, lo que podría 
    llevar a un mayor estrés, fatiga y riesgo de accidentes laborales. Desde el punto de vista del 
    utilitarismo, esta opción sería éticamente cuestionable ya que, aunque podría aumentar la 
    eficiencia y la rentabilidad para los empleadores, disminuiría el bienestar de los trabajadores 
    y, por lo tanto, no produciría la mayor cantidad de felicidad o bienestar general.

    \item \textbf{Emplear el algoritmo con restricciones:} En este caso, se establecerán restricciones 
    en el algoritmo para garantizar que se respeten los derechos y el bienestar de los trabajadores, 
    incluso si esto disminuye la eficiencia o rentabilidad. Por ejemplo, se podría establecer que los 
    trabajadores no trabajen más allá de un cierto número de horas al día, que se les otorgue tiempo 
    suficiente para descansar y que se les brinde un entorno de trabajo seguro. Esta opción sería 
    éticamente preferible, ya que se busca maximizar tanto la eficiencia como el bienestar de los 
    trabajadores, lo que llevaría a una mayor felicidad general.

    \item \textbf{No emplear el algoritmo:} Esta opción implicaría que la empresa continúe 
    programando los trabajos de manera manual, sin utilizar el algoritmo. Desde el punto de vista 
    ético sería una mala decisión ya que una mala distribución de tareas derivada de no utilizar 
    herramientas como la aquí mencionada podría resultar en una menor eficiencia en el trabajo 
    y en la productividad de la empresa, lo que a su vez podría afectar negativamente a sus 
    empleados y clientes. Sin embargo, si la implementación del algoritmo no produce una mejora 
    significativa en la eficiencia o si esta mejora no es suficiente para justificar los posibles 
    costos en el bienestar de los trabajadores, entonces esta opción podría ser éticamente preferible.
\end{enumerate}

\subsubsection{Desde la perspectiva de la ética de la virtud}
Desde esta perspectiva nos preguntamos cómo la resolución del problema podría afectar la 
integridad y el carácter de los trabajadores de la empresa. Si la solución del problema 
implicara la violación de valores importantes entonces podría considerarse una acción inmoral 
desde la perspectiva de la ética de la virtud. Vamos a estudiar como afectan a las personas 
diferentes aplicaciones del algoritmo teniendo en cuenta algunos criterios.

\begin{itemize}
    \item \textbf{Priorizar la maximización del beneficio:} Esta opción puede parecer 
    atractiva en el corto plazo, pero a largo plazo puede conducir a la explotación de 
    los trabajadores y a la pérdida de la confianza de los clientes y la comunidad en 
    general. Desde una perspectiva ética aristotélica, esta opción no es virtuosa, ya 
    que no tiene en cuenta la felicidad de los trabajadores y no contribuye al bienestar 
    de la comunidad en general.
    \item \textbf{Buscar un equilibrio entre el bienestar de los trabajadores y 
    la rentabilidad de la organización:} Esta opción puede parecer la más equilibrada desde 
    la perspectiva de la ética aristotélica. Al buscar un equilibrio entre los intereses de 
    los trabajadores y los de la organización, se fomenta tanto la justicia como la prudencia. 
    Esta opción también puede conducir a una mayor estabilidad financiera y al desarrollo de 
    relaciones más confiables con los trabajadores mejorando la comunidad en general.
    \item \textbf{Mantener el estado actual de los trabajadores:} Esta opción puede parecer 
    noble y virtuosa, pero también puede llevar a un estado financiero insostenible. Desde 
    una perspectiva ética aristotélica, es importante tener en cuenta tanto la justicia como 
    la prudencia en la toma de decisiones que afectan tanto a los trabajadores como a la 
    organización en general. Si se sabe que el algoritmo puede mejorar las condiciones de 
    trabajo y mejorar la producción, podría darse el caso de que no el utilizar el algoritmo 
    supiese un retraso de la empresa frente a sus competidores, tal vez incluso suponiendo el 
    quiebre de la empresa y la desaparición de multitud de puestos de trabajo.
\end{itemize}

\subsubsection{Bajo la ética principialista Kantiana} 
La corriente ética principialista kantiana se basa en la idea de que las acciones deben 
ser juzgadas por la intención detrás de ellas, y no solo por sus consecuencias. Además, 
se sostiene que las personas deben ser tratadas siempre como fines en sí mismas, y no 
como medios para lograr fines externos. Aquí se identifican diferentes principios que 
podrían ser afectados al intentar aplicar este algoritmo en una situación real. 

\begin{itemize}
    \item \textbf{Principio de autonomía:} se refiere a la capacidad de 
    las personas para tomar sus propias decisiones y actuar de acuerdo con sus propias 
    creencias y valores. En este caso, el algoritmo podría afectar la autonomía de los 
    trabajadores al asignarles tareas sin su consentimiento y sin tener en cuenta sus
    preferencias. 
    \item \textbf{Principio de beneficencia:} se refiere a la obligación 
    de las personas de ayudar a los demás. En este caso, el algoritmo podría afectar
    negativamente a los trabajadores al suprimir trabajos que podrían ser realizados
    por ellos, y que podrían ser beneficiosos para ellos.

    \item \textbf{Principio de justicia:} se refiere a la obligación de las
    personas de tratar a los demás de manera justa. En este caso, el algoritmo podría afectar
    la justicia de los trabajadores al asignarles tareas que no son capaces de realizar o que
    no desean realizar.

    \item \textbf{Principio de privacidad:} La IA debe salvaguardar la privacidad y proteger 
    la confidencialidad de los datos utilizados en el FJSP, respetando las normativas y 
    regulaciones de protección de datos.

    \item \textbf{Sostenibilidad social:} el desarrollo de la IA debe considerar su 
    impacto en la sociedad a largo plazo, fomentando beneficios a largo plazo y evitando 
    consecuencias negativas en términos de empleo, desigualdad o exclusión social.

\end{itemize}

\subsection{Conclusión ética}
En este apartado, se darán diferentes opciones derivadas del método de análisis, 
se discutirá la elección final tomada y se darán las razones detrás de esa elección. 
Se examinará cómo se llegó a la elección final y se explicará por qué se consideró 
la mejor opción en ese momento.

\begin{itemize}
    \item Opción 1, aplicar el algoritmo sin tener en cuenta ningún tipo de consideración ética y ateniéndose únicamente a los beneficios de la producción. Esta opción se enfoca únicamente en los beneficios de la producción, sin considerar el impacto en los trabajadores o la ética involucrada en la implementación del algoritmo. Esto podría llevar a una reducción de costos a corto plazo y un aumento de la eficiencia, pero a largo plazo afecta negativamente la moral y el bienestar de los trabajadores. Esta opción queda completamente descartada.
    \item Opción 2, no aplicar el algoritmo para mantener el estado actual de la empresa y no aplicar ninguna mejora. La segunda opción implica no hacer ningún cambio en la empresa, lo que puede hacer que la empresa se quede atrás en términos de eficiencia y competitividad. Además, esto puede llevar a la insatisfacción de los empleados, ya que no se están implementando mejoras que podrían hacer que su trabajo sea más fácil y eficiente. Por lo tanto, esta opción puede no ser sostenible a largo plazo y puede llevar a la disminución de la productividad y la rentabilidad de la empresa. Aun así, es posible aplicarla en situaciones donde el aplicar el algoritmo no siga las consideraciones previamente mencionadas.
    \item Opción 3, aplicar el algoritmo de forma equilibrada, esto es teniendo en cuenta las necesidades y preocupaciones de los trabajadores a su vez con una mejora del proceso productivo: La última opción busca un equilibrio entre la eficiencia y las condiciones de los trabajadores. Si se logra este equilibrio, podría aumentar la productividad mientras se asegura el bienestar de los trabajadores. Esta opción es la escogida finalmente, ya que aporta un beneficio tanto a los trabajadores como a la productividad de la empresa en comparación al estado inicial, la idea del equilibrio es que todos se vean beneficiados de este cambio y no se vulneren la integridad de los trabajadores.
\end{itemize}

\subsubsection{Test de convergencia / divergencia}
En términos generales las diferencias corrientes convergen de forma uniforme dentro de la 
opción 3 y parece ser la elección final más adecuada y justa para todas las partes involucradas, 
ya que esta opción busca un equilibrio entre la eficiencia y la ética, tomando en cuenta las 
necesidades y preocupaciones de los trabajadores. Además, se refleja la opción 1 como cuestionable 
éticamente o incluso descartable debido a que todos los planteamientos nos llevan a que no se puede 
anteponer la producción a los empleados, mientras que la opción 2 ,aunque sí que puede plantear 
inquietudes en ciertos aspectos, existe mayor divergencia ya que dependiendo de qué corriente 
se trate puede tener mayor o menor impacto en la decisión final.\medskip

Se ha considerado como la mejor opción la tercera, pero antes de aplicarla será necesario 
evaluar la situación de la empresa e identificar claramente en qué lugares y bajo qué 
condiciones será su puesta en producción. Sólo si todas estas condiciones son satisfechas 
nos encontraremos ante un caso donde se pueda aplicar la opción 3, en otros casos se derivará 
a la opción 2 de no aplicar el algoritmo.
