\section{Definiciones, Acrónimos y Abreviaturas}
\subsection{Acrónimos y Abreviaturas}
\begin{itemize}
    \item \textbf{FJSP: } Flexible Job Shop Scheduling Problem.
    \item \textbf{JSP: } Job Scheduling Problem.
    \item \textbf{IA: } Inteligencia Artificial.
    \item \textbf{ML: } Machine Learning.
    \item \textbf{RL: } Reinforcement Learning.
    \item \textbf{IL: } Imitation Learning.
    \item \textbf{SPT:} Shortest Processing Time
    \item \textbf{LRW:} Longest Work Remaining
    \item \textbf{MWR:} Most Work Remaining
    \item \textbf{CSV: } Comma Separated Values.
\end{itemize}

\subsection{Definiciones}
\begin{itemize}
    \item \textbf{Acción:} una acción es una decisión que toma el agente y que tiene una
    repercusión sobre estado del environment.
    \item \textbf{Agente:} el agente es el encargado de interactuar con el environment
    y realizar acciones sobre él. El agente aprende de las recompensas que recibe
    en función de las decisiones que toma y de esta forma mejorar su comportamiento.
    \item \textbf{Benchmark:} un benchmark es un conjunto de problemas que se utilizan
    para evaluar el rendimiento de un algoritmo.
    \item \textbf{Entrenamiento:} el entrenamiento es el proceso de aprendizaje de un
    modelo de ML. Durante el entrenamiento, el modelo aprende de los datos para realizar
    una tarea específica.
    \item \textbf{Environment: } un environment es un entorno donde se ejecuta un agente
    de RL. El environment proporciona al agente información sobre el estado del entorno
    y el agente puede realizar acciones que afectan al environment.
    \item \textbf{Episodio:} un episodio es una secuencia de pasos que realiza el agente
    en el environment. Un episodio comienza cuando el agente se encuentra en un estado
    inicial y termina cuando el agente alcanza un estado final.
    \item \textbf{Estado:} el estado es la información que el environment proporciona
    al agente sobre el entorno. El estado puede ser parcial o completo, dependiendo
    de si el agente tiene acceso a toda la información del entorno o solo a una parte.
    \item \textbf{Formateador:} un formateador es una herramienta que analiza el código fuente
    para formatearlo de forma que cumpla con un estilo de código concreto.
    \item \textbf{Heurística:} una heurística es una técnica que se utiliza para resolver
    un problema de forma aproximada. Las heurísticas no garantizan encontrar la solución
    óptima, pero suelen encontrar soluciones aceptables en un tiempo razonable.
    \item \textbf{Hiperparámetro:} los hiperparámetros son parámetros que se utilizan
    para configurar el modelo de ML. Estos parámetros no se aprenden durante el entrenamiento
    del modelo, sino que se establecen antes de comenzar el entrenamiento.
    \item \textbf{Linter:} un linter es una herramienta que analiza el código fuente 
    para señalar errores de programación, errores de estilo, construcciones sospechosas, etc.
    \item \textbf{Metodología ágil:} las metodologías ágiles son metodologías de desarrollo
    de software que se basan en el desarrollo iterativo e incremental. Las metodologías ágiles
    se centran en la entrega de software funcional en periodos cortos de tiempo.
    \item \textbf{Metaheurística:} una metaheurística son algoritmos de búsqueda inteligentes 
    que se inspiran en procesos naturales o en conceptos matemáticos abstractos, y son capaces 
    de explorar y explotar el espacio de soluciones en busca de soluciones óptimas o de alta calidad.
    \item \textbf{Modelo:} un modelo de ML es un algoritmo que aprende de los datos
    para realizar una tarea específica.
    \item \textbf{Profiling:} el profiling es una técnica que se utiliza para analizar
    el rendimiento de un programa con el objetivo de encontrar fallos de rendimiento.
    \item \textbf{Pipeline: } un pipeline es una secuencia de pasos que se ejecutan
    de forma secuencial para realizar una tarea.
    \item \textbf{Recompensa:} la recompensa es la información que el environment
    proporciona al agente sobre la calidad de las acciones que ha realizado.
    \item \textbf{Redes Neuronales:} las redes neuronales son un modelo de ML inspirado
    en el funcionamiento del cerebro humano. Las redes neuronales están formadas por
    neuronas artificiales que se organizan en capas y se conectan entre sí.
    \item \textbf{Reinforcement Learning:} el RL es un paradigma de ML que se basa
    en el aprendizaje por refuerzo. El agente aprende de las recompensas que recibe
    en función de las acciones que realiza y de esta forma mejorar su comportamiento.
    \item \textbf{SCRUM:} SCRUM es una metodología ágil para la gestión y planificación
    de proyectos de software.
\end{itemize}

\pagebreak