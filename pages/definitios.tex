\section{Definiciones, Acrónimos y Abreviaturas}
\subsection{Acrónimos y Abreviaturas}
\begin{itemize}
    \item \textbf{FJSP: } Flexible Job Shop Scheduling Problem.
    \item \textbf{JSP: } Job Scheduling Problem.
    \item \textbf{IA: } Inteligencia Artificial.
    \item \textbf{RL: } Reinforcement Learning.
    \item \textbf{IL: } Imitation Learning.
    \item \textbf{CSV: } Comma Separated Values.
\end{itemize}

\subsection{Definiciones}
\begin{itemize}
    \item \textbf{Linter: } Un linter es una herramienta que analiza el código fuente 
    para señalar errores de programación, errores de estilo, construcciones sospechosas, etc.
    \item \textbf{Formateador: } Un formateador es una herramienta que analiza el código fuente
    para formatearlo de forma que cumpla con un estilo de código concreto.
    \item \textbf{Modelo: } Un modelo de ML es un algoritmo que aprende de los datos
    para realizar una tarea específica. 
    \item \textbf{Environment: } Un environment es un entorno donde se ejecuta un agente
    de RL. El environment proporciona al agente información sobre el estado del entorno
    y el agente puede realizar acciones que afectan al environment.
\end{itemize}

\pagebreak