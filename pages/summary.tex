\section*{Resumen}
El propósito del siguiente proyecto es investigar y desarrollar una nueva solución para el
problema de optimización “flexible job shop scheduling problem” (FJSP), que se clasifica dentro
de la categoría de problemas NP-hard, que son aquellos que no pueden ser resultados de forma
exacta por un algoritmo en un tiempo de computación polinómica. El objetivo es asignar una
serie de operaciones a un conjunto de máquinas, teniendo en cuenta una lista de restricciones
y requisitos, como por ejemplo el tiempo de procesamiento. Por último, como resultado de lo
anterior, se proveerá de una configuración válida, que intente minimizar el tiempo máximo de
finalización (make span) de todas las máquinas.\medskip

Se propone utilizar técnicas de Imitation Learning, una variante del
Reinforcement learning que permite al modelo aprender a partir de ejemplos
proporcionados por un experto. En nuestro caso, se estudiarán técnicas de
generación de ejemplos válidos que permitan al modelo aprender técnicas para
resolver el problema. La idea, es extraer configuraciones óptimas de instancias
pequeñas generadas aleatoriamente y utilizarlas como datos de entrenamiento
para un modelo, que pueda desarrollar una estrategia propia en base a las
decisiones del experto y posteriormente generalizar a instancias
mayores.\medskip

El objetivo es mejorar la velocidad en la que se ofrecen planificaciones
cercanas al óptimo, ya que con este planteamiento no será necesario explorar
todo el espacio de soluciones. Otro de los beneficios es la adaptabilidad a
variaciones del problema dinámicamente, como cambios en los tiempos de
procesamiento o la incorporación de nuevas operaciones. Teniendo esto en
cuenta, se espera obtener una solución en tiempo lineal que no se desvíe
excesivamente del óptimo y que pueda aplicarse a diferentes escenarios
industriales e incluso en consonancia con otras técnicas.

\section*{Descriptores}
Reinforcement Learning, Imitation Learning, FJSP, Redes Neuronales, MLOPS
\pagebreak