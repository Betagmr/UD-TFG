\subsection{Metodología de investigación}
En esta sección, se describirá la metodología desarrollada para abordar problemas 
dentro del marco de trabajo de los problema NP-Hard. La metodología propuesta se 
basa en el uso de técnicas de IL para abordar problemas computacionalmente
complejos, dando resultados cercanos al optimo en un tiempo de ejecución muy reducido. 
Se explicará cómo esta metodología se ha diseñado y aplicado para resolver el caso concreto 
del FJSP, pero como se ha mencionado anteriormente, esta metodología se podría aplicar a otros 
problema de optimización combinatoria NP-Hard.\medskip

Una vez que sea descrita la metodología para abordar el problema utilizando 
técnicas de IL, se procederá con el caso especifico de cómo esta metodología 
ha sido aplicada para diseñar un modelo del FJSP. Se explicará cómo se ha procedido en cada uno 
de los pasos propuestos a nivel de razonamiento y ejecución. A continuación se van a listar una 
serie de puntos que se deben seguir para poder aplicar la metodología propuesta:

\begin{enumerate}
    \item \textbf{Modelado del problema como un environment de RL.} El problema 
    se modeliza como un environment de RL, definiendo el espacio de estados 
    y el espacio de acciones disponibles para el agente. A diferencia de un enfoque típico de RL, 
    no es necesario definir una función de recompensa ya que el modelo se entrenará mediante aprendizaje 
    supervisado y no por refuerzo. Es importante destacar que el espacio de estados y el espacio de
    acciones estén enfocados en construir gradualmente la solución del problema y ser capaces
    de representar una solución valida.
    \item \textbf{Generación de instancias aleatorias para el entrenamiento del modelo.} 
    Se generan instancias aleatorias del problema con el fin de crear un conjuntos de datos 
    para el modelo de aprendizaje automático. Estas instancias aleatorias se utilizaran para entrenar el 
    modelo de deep learning, que servirá para predecir soluciones en instancias del problema no vistas 
    previamente. Una recomendación es que adicionalmente se generen diferentes grupos de instancias,
    con diferentes tamaños y características, para poder evaluar el rendimiento del modelo en diferentes
    escenarios.
    \item \textbf{Utilización de métodos de programación matemática para obtener la solución óptima de 
    las instancias.} Mediante métodos de programación matemática, se busca obtener las soluciones óptimas 
    para las instancias generadas en el paso anterior. Estas soluciones óptimas se utilizaran como los 
    datos del experto, lo que permitirá al modelo desarrollar una estrategia para resolver instancias no vistas.
    \item \textbf{Entrenamiento del modelo supervisado de deep learning.} Una vez que se tienen las instancias
    generadas y las soluciones óptimas, se procesarán a traves del environment de tal forma que para cada
    instancia se obtenga una secuencia de acciones que representan la solución óptima. Estas secuencias de
    acciones serán las que el modelo de deep learning tratará de predecir.
    \item \textbf{Evaluación de resultados y benchmark.} Se utilizó el modelo entrenado 
    en el paso anterior para predecir soluciones en instancias del problema que forman parte del benchmark 
    o conjunto de datos de prueba. El objetivo es evaluar la capacidad del modelo para generalizar y 
    proporcionar soluciones adecuadas en instancias no vistas previamente.
\end{enumerate}

