\section{Objetivos y alcance}
En esta sección se introducen los objetivos en los que va consistir el proyecto, habiendose
realizado una division entre el principal y los secundarios. Ademas de ello se presentan los elementos
que forman el alcance, asi como se comentaran otros que no forman parte de este.  

\subsection{Objetivos generales}
El objetivo del proyecto es desarrollar un modelo de IA que optimice la asignación 
de tareas a máquinas en una línea de producción, minimizando el tiempo total de 
producción y maximizando la utilización de los recursos disponibles, todo ello 
a partir de ejemplos de configuraciones proporcionados por un experto. El modelo 
deberá desarrollar una estrategia propia a partir de los ejemplos de entrenamiento
que le permita tomar decisiones en situaciones nuevas en un tiempo de ejecución
reducido. Para ello, a countinuación se describen los objetivos específicos:

\begin{itemize}
\item \textbf{Identificar al experto:} el primer objetivo consta de seleccionar un
experto capaz de generar una muestra representativa de las configuraciones y decisiones 
optimas a realizar en diferentes situaciones. Estos ejemplos servirán como datos 
de entrenamiento para el modelo de inteligencia artificial.
\item \textbf{Modelar el problema:} buscar una representación adecuada que permita gestionar 
las configuraciones y decisiones tomadas por el experto. Al tratarse de un modelo basado en 
Imitation Learning, es necesario definir un envioronment que permita la interacción entre el 
modelo y el estado del problema.
\item \textbf{Entrenar el modelo:} utilizando los ejemplos recopilados, se debe entrenar 
el modelo de inteligencia artificial para que pueda aprender a imitar el comportamiento 
del experto en la toma de decisiones. Esto implica ajustar los parámetros del modelo para 
minimizar el error entre las salidas predichas por el modelo y las decisiones tomadas por 
el experto en los ejemplos de entrenamiento.
\item \textbf{Evaluar el modelo:} comprobar que el modelo ha inferido correctamente
los patrones de comportamiento del experto, se debe evaluar su rendimiento en situaciones
nuevas. Para ello, se utilizarán instancias de prueba que no han sido utilizadas en el
entrenamiento del modelo. Además, se estudiarán diferentes métricas que permitan evaluar 
correctamente el desempeño del modelo. Por último, se tendran en cuenta aspectos adicionales 
como el tiempo de ejecución del modelo.
\item \textbf{Optimizar el modelo:} cuando el desempeño del modelo no es satisfactorio o 
se intuyen posibles mejoras, se deben identificar que áreas el modelo muestra dificultades y realizar ajustes 
en su arquitectura, parámetros o datos de entrenamiento para mejorar su rendimiento. Esto 
se puede lograr mediante el aumento del volumen del entrenamieto, monitorizando el desempeño
del mismo o mediante la optimización de los hiperparámetros de entrenamiento.
\item \textbf{Implementar el modelo:} por último, se debe implementar el modelo dentro de
una aplicación que permita su uso por parte de los usuarios. Esta aplicación debe ser capaz
de cargar iferentes configuraciones del problema, ejecutar el modelo y graficar los resultados
obtenidos.
\end{itemize}

\subsection{Alcance}
En esta sección se definen los límites del proyecto, estableciendo lo que está incluido 
y excluido dentro mismo. Se describira de manera detallada las actividades que forman 
parte del desarrollo final, así como aquellos elementos que no están incluidos en el 
alcance del proyecto. Esta sección es fundamental para tener una comprensión clara 
y evitar posibles malentendidos o confusiones durante su ejecución.

\subsubsection{Dentro del alcance}
\begin{itemize}
    \item \textbf{Variante del problema:} la versión del problema que se abordará en este proyecto
    es una variante del Job Shop Scheduling Problem. Este problema se caracteriza por una mayor
    complejidad debido a que una misma tarea puede ser ejecutada por diferentes máquinas.
    \item \textbf{Generación de ejemplos de entrenamiento:} el set de entrenamiento estará compuesto
    por instancias generadas de forma aleatoria, estas instancias serán generadas sintéticamente 
    utilizando un algoritmo que crea instancias de problemas FJSP con tiempos de preparación. 
    \item \textbf{Análisis de resultados mediante benchmarks:} se utilizarán benchmarks
    especificos para medir el rendimiento del modelo. Estos benchmarks son conjuntos de instancias
    especialmente dificiles, que reflejan de manera objetiva el desempeño del modelo.
    \item \textbf{Comparación contra tecnicas actuales:} durante las diferentes fases del desarrollo,
    se monitorizara el rendimiento del modelo para compararlo posteriormente con las técnicas actuales 
    que se emplean en el estado del arte para la resolución del problema. 
    \item \textbf{Desarrollo de una aplicacion web sencilla:} la aplicación final contará con una interfaz 
    web que permita visualizar los resultados obtenidos por el modelo. Esta aplicación
    permitirá cargar diferentes configuraciones del problema, ejecutar el modelo y visualizar
    un orgnaigrama con las tareas asignadas a cada máquina.
\end{itemize}

\subsubsection{Fuera del alcance}
\begin{itemize}
    \item \textbf{Recopilación de datos en tiempo real:} no se incluirá la recopilación de datos en tiempo 
    real durante la ejecución del modelo. Se utilizarán ejemplos de configuraciones proporcionados 
    por el experto como datos de entrenamiento, pero no se recopilarán datos de decisiones tomadas 
    en tiempo real durante la implementación del modelo.
    \item \textbf{Cambio de configuración durante la ejecución:} no se contempla la posibilidad de cambiar
    la configuración del problema durante la ejecución del modelo. Aunque se podría implementar debido
    a que el modelo sera capaz de adaptarse a cambios en la configuración, se deberia realizar un 
    trabajo extra dentro de la aplicación web para que el usuario pueda cambiar la configuración.
\end{itemize}

\pagebreak