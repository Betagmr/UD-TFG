\subsection{Incidencias del Proyecto}
En esta sección del TFG, se aborda uno de los aspectos más importantes del desarrollo 
de un proyecto de Machine Learning: los problemas y desafíos que pueden surgir durante 
el proceso. Es común que se presenten dificultades que pueden afectar tanto la precisión 
como la eficiencia del modelo. Estos problemas pueden ser causados por una variedad 
de factores a continuación se describen algunos de los inconvenientes que se presentaron
durante el desarrollo del proyecto.

\subsubsection{Problemas con la normalización de los datos}
La normalización de los datos de entrada es un paso crítico para garantizar que el 
modelo pueda aprender de manera efectiva y generar predicciones precisas. En el caso 
específico de los grafos, el estado de un nodo puede almacenar diferentes tipos de datos, 
como números, referencias a otros nodos, etc, lo que puede generar problemas a la hora 
de la normalización. Otro lugar done también surgen problemas de normalización es en los 
datos de las aristas, que representan las conexiones entre nodos. Al respresentar un 
nodo como una lista de características, estas diferentes caracteristicas de pueden 
requerir de diferentes técnicas de normalización.\medskip

La soluación que se ha adoptado en este proyecto es la de normalizar los datos de
cada entrada de manera independiente. De esta manera, se evitan problemas de colision 
entre los diferentes nodos y aristas de un mismo grafo. Esta solución, sin embargo,
trae problemas de escalabilidad ya que cada vez que se implementa y/o añade una nueva
característica al modelo, es necesario modificar el código de normalización de los datos.\medskip

En un futuro, se podría implementar una solución más escalable, que permita definir
un conjunto de tipos de características con sus respectivas técnicas de normalización.
De este modo, se podría añadir nuevas características al modelo sin necesidad de
modificar el código de normalización. Además, tambien daria la posibilidad de
poder en una fase de testeo, probar diferentes técnicas de normalización para
cada tipo de característica y elegir la que mejor se adapte al modelo.

\subsubsection{Problemas con el hardware de entrenamiento}
La falta de hardware adecuado para el entrenamiento y testeo es un problema común 
que puede afectar la eficiencia y la precisión del modelo. El entrenamiento de modelos 
de Machine Learning a menudo requiere una gran cantidad de recursos computacionales, 
como procesadores, memoria RAM y capacidad de almacenamiento. Además, los modelos 
pueden requerir GPUs (unidades de procesamiento gráfico) especializadas para acelerar 
el proceso de entrenamiento.\medskip

La falta de hardware adecuado puede limitar la cantidad de datos que se pueden usar 
para entrenar el modelo, la complejidad del modelo que se puede entrenar y la 
velocidad a la que se puede realizar el entrenamiento. Esto puede resultar en 
modelos que no sean lo suficientemente precisos o que no se puedan entrenar en 
un tiempo razonable. En el caso de este proyecto, se ha contado con un pequeño
servidor que disponia de una CPU y 4GB de RAM, lo que ha limitado el tamaño de los
grafos que se han podido usar para el entrenamiento. Además, el entrenamiento de
los modelos y la búsqueda de hiperparámetros ha sido un proceso lento, ya que
no se ha contado con una GPU para acelerar el proceso.\medskip

En un futuro, se podría implementar una solución que permita entrenar los modelos
utilizando una o varias GPUs. Esto permitiría entrenar modelos más complejos y
con un mayor número de datos y daría la posibilidad de aumentar el número de
caracteristicas de los nodos y aristas de los grafos.
