\subsection{Tecnologías Utilizadas}
En la sección se presentarán todas las tecnologías que han tenido un impacto directo en el desarrollo de las 
funcionalidades del proyecto de Machine Learning.

\subsubsection{Python}
Python es un lenguaje de programación interpretado de alto nivel. Es una de las herramientas más populares 
en la comunidad de Data Science y Machine Learning debido a su facilidad de uso, su amplia gama de bibliotecas 
y su enfoque en la legibilidad del código. Además, recientemente se está haciendo una tarea de mejora del 
lenguaje con la inclusión de nuevas librerías basadas en C para un mejor rendimiento, inclusión de elementos 
de tipado estático para el mantenimiento del código y mejoras en el sistema de errores.

\subsubsection{NumPy}
NumPy es una biblioteca de Python utilizada para trabajar con matrices y arrays multidimensionales de manera 
eficiente. Fue desarrollada en C++ para ser utilizada en aplicaciones de Data Science y matemáticas, ofrece 
una amplia variedad de funciones y herramientas para operar con matrices y realizar cálculos numéricos complejos. 
Además, al estar escrito en C++ está altamente optimizado para trabajar con grandes conjuntos de datos, realizar 
operaciones aritméticas y lógicas, de indexar y redimensionar matrices de manera eficiente, realizar transformaciones 
de Fourier y realizar cálculos de álgebra lineal.

\subsubsection{Pandas}
Pandas es una biblioteca de Python utilizada para la manipulación y análisis de datos. Es especialmente 
útil para el manejo de grandes conjuntos de datos y ofrece una variedad de herramientas y funciones para 
trabajar con diferentes tipos de datos y formatos. Una de las estructuras de datos principales de Pandas es 
el DataFrame, que se puede considerar como una tabla de datos bidimensional, donde cada fila representa una 
observación y cada columna representa una variable. Los DataFrames permiten la manipulación y transformación 
de datos mediante una variedad de operaciones, lo que los hace muy útiles en aplicaciones de ciencia de datos 
y análisis de datos. 

\subsubsection{PyTorch}
PyTorch es un framework de código abierto de Deep Learning hecho en Python, desarrollado por el equipo 
de Inteligencia Artificial de Facebook. Se utiliza para crear y entrenar modelos de redes neuronales para resolver 
problemas de Machine learning. PyTorch es muy popular en la comunidad de investigación debido a su facilidad de uso 
y su capacidad para realizar cálculos en GPU para acelerar el proceso de entrenamiento de modelos. Además, PyTorch
ofrece una gran variedad de herramientas y funciones para trabajar con redes neuronales, como la creación de
diferentes tipos de redes neuronales, la definición de funciones de pérdida, la definición de optimizadores, la 
definición de funciones de activación, etc.

\subsubsection{Gym}
Gym es una biblioteca de Python utilizada para el diseño y creación de environments de RL. Incluye un 
gran conjunto de herramientas para trabajar con algoritmos personalizados y ofrecre muchas facilidades en 
el registro y de monitoreo del proceso de entrenamiento. Todas estas razones lo convierte en la opción más popular 
para los desarrolladores cuando abordan problemas de RL.

\subsubsection{PyTorch Geometric}
PyTorch Geometrics es una libreria que ofrece soporte para el procesamiento de grafos, lo que significa 
que se puede utilizar para trabajar con redes de datos en las que los nodos están conectados por enlaces. 
Esto puede ser útil ya que mediante técnicas de Deep Learning se puede analiza la estructura y las 
relaciones que existen entre los nodos de un grafo y extraer de ahi sus caracteristicas. Además, la 
librería ofrece herramientas para trabajar con diferentes tipos de grafos, como grafos dirigidos y heterogéneos, 
que son los utilizados en el proyecto, y proporciona una serie de funciones para realizar 
operaciones básicas con grafos.

\subsubsection{Or-tools}
OR-Tools es una biblioteca de optimización combinatoria de código abierto para Python y otros 
lenguajes de programación. Permite resolver problemas de optimización de manera eficiente, utilizando una 
amplia variedad de algoritmos y técnicas de programación matemática. Con OR-Tools, los usuarios pueden 
resolver problemas de programación lineal, programación entera mixta, programación cuadrática, 
problemas de rutas y muchos otros. Además, la biblioteca también incluye herramientas para resolver 
problemas de programación con restricciones, como la asignación de tareas o la programación de horarios.

\subsubsection{Optuna}
Optuna es una biblioteca de optimización de hiperparámetros de código abierto para Python. Su objetivo 
es automatizar el proceso de ajuste de hiperparámetros, que a menudo es un proceso intensivo en 
recursos y muy demandante en tiempo, para permitir a los usuarios encontrar la mejor configuración 
de hiperparámetros de sus modelos de Machine Learning de manera eficiente. Optuna utiliza algoritmos 
de optimización bayesiana para buscar los mejores valores de hiperparámetros en función del 
rendimiento de los modelos en el conjunto de validación. Con Optuna, los usuarios pueden definir una función 
objetivo y especificar los rangos de búsqueda de cada hiperparámetro, así como las restricciones entre ellos, 
lo que permite a la biblioteca ajustar varios hiperparámetros al mismo tiempo.

