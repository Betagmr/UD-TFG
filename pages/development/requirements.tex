\subsection{Especificación de Requisitos del Sistema}

\subsubsection{Stakeholders}
Dado que este proyecto es de indole de academica sobre como aplicar 
tecnicas de IL en problemas NP-Hard no tiene clientes directos. Sin embargo, 
teniendo en cuenta hacia qué está enfocado y la utilidad del mismo, se pueden 
identificar 2 tipos de grupos de interes. En primer lugar, los investigadores 
dentro de Tecnalia, ya que un avance en esta problematica les puede servir como punto de partida
para mejorar sus propios algoritmos. En segundo lugar, los clientes de Tecnalia,
ya que podrán utilizar esta herramienta para resolver sus propios problemas de
optimización con unos pequeños ajustes.

\subsubsection{Restricciones Obligatorias}
\begin{itemize}
    \item \textbf{Limitación de tecnologías:} el proyecto esta limitado al uso de
    tecnologías y/o herramientas de código abierto o con licencias apropiadas.
    \item \textbf{Limitación de recursos:} el proyecto esta limitado 
    por recursos como el tiempo, el presupuesto, la disponibilidad de personal, 
    el equipo de hardware y software.
    \item \textbf{Requisitos legales y éticos:} el proyecto esta limitado 
    por los requisitos legales y éticos que deben cumplirse, por ejemplo, en lo 
    que respecta a la privacidad de los datos, la seguridad, la propiedad intelectual 
    y la ética en la investigación.
    \item \textbf{Limitación de acceso a datos:} el proyecto esta limitado
    por el acceso a los datos de los clientes de Tecnalia. En este caso, se
    utilizarán datos generados sinteticamente para poder realizar el proyecto.
\end{itemize}

\subsubsection{Reglas de Negocio}
\begin{itemize}
    \item Al aplicar la metodologia se deben incluir reglas 
    de validación para garantizar que los datos de entrada sean coherentes 
    y estén dentro de los límites aceptables para el problema a resolver.
    \item Se deben establecer los parámetros y procedimientos para el proceso 
    de entrenamiento del modelo de aprendizaje automático que se utilizará 
    para la resolución del problema, incluyendo los criterios de evaluación del modelo.
    \item Se deben especificar las reglas para la interpretación de los resultados 
    obtenidos por la metodología, así como las medidas de calidad que se utilizarán 
    para evaluar el rendimiento de la solución propuesta.
    \item La metodología debe incluir reglas para la gestión de errores
    de la metodología, incluyendo la revisión periódica de las reglas de negocio 
    y la actualización de los modelos de aprendizaje automático.
\end{itemize}
\subsubsection{Catálogo de Requisitos Funcionales}
\begin{enumerate}
    \renewcommand{\labelenumi}{RF\arabic{enumi}}
    \item Se debe proporcionar una definicion clara de los objetivos de la investigación 
    y cómo se van a alcanzar.
    \item Aplicando la metodología debe ser capaz de generar el conjunto de datos de entrenamiento
    y validacion a partir de sistemas de generación de instancias aleatorias.
    \item La metodología debe idenficar diferentes formas para encontrar un experto
    que pueda generar soliciones optimas del caso de uso del problema.
    \item La metodología debe incluir el diseño y entrenamiento de un modelo de Machine Learning 
    adecuado para la tarea específica, de manera que se optimice su precisión y eficiencia.
    \item Aplicando la metodología debe ser capaz de medir y evaluar el rendimiento del modelo 
    utilizando una o varias metricas de calidad específicas para el caso en concreto.
    \item La metodología debe proporcionar un marco para el diseño de la investigación, 
    incluyendo la definición de variables, la identificación de las fuentes de datos y 
    la elección de los métodos de recolección de datos.
    \item La metodología debe incluir la interpretación de los resultados del modelo 
    de Machine Learning, de manera que se puedan extraer conclusiones útiles y relevantes.
\end{enumerate}

\subsubsection{Requisitos no Funcionales}
\begin{enumerate}
    \renewcommand{\labelenumi}{RNF\arabic{enumi}}
    \item La metodología debe ser fácil de entender y utilizar tanto para 
    investigadores expertos como para aquellos que no tienen experiencia en el campo.
    \item La metodología debe ser reproducible por otros investigadores, de manera 
    que los resultados obtenidos puedan ser comparados y validados.
    \item La metodología debe ser lo suficientemente flexible para adaptarse a 
    diferentes situaciones y contextos de investigación.
    \item La metodología debe ser clara y precisa en su presentación y explicación, 
    para evitar ambigüedades y malinterpretaciones.
    \item La metodología debe ser innovadora y estar actualizada con los últimos 
    avances en investigación y tecnología. 
    \item Al desrrolar la metodología se deben seguir los principios de código limpio, 
    utiliar una arquitectura adecuadas y patrones de diseños cuando sea necesario.
\end{enumerate}