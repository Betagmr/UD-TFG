\subsection{Especificación de Requisitos del Sistema}

\subsubsection{Stakeholders}
Dado que este proyecto es de indole de academica sobre como aplicar 
tecnicas de IL en problemas NP-Hard no tiene clientes directos. Sin embargo, 
teniendo en cuenta hacia qué está enfocado y la utilidad del mismo, se pueden 
identificar 2 tipos de grupos de interes. En primer lugar, los investigadores 
dentro de Tecnalia, ya que un avance en esta problematica les puede servir como punto de partida
para mejorar sus propios algoritmos. En segundo lugar, los clientes de Tecnalia,
ya que podrán utilizar esta herramienta para resolver sus propios problemas de
optimización con unos pequeños ajustes.

\subsubsection{Restricciones Obligatorias}
\begin{itemize}
    \item \textbf{Limitación de tecnologías:} el proyecto esta limitado al uso de
    tecnologías y/o herramientas de código abierto o con licencias apropiadas.
    \item \textbf{Limitación de recursos:} el proyecto esta limitado 
    por recursos como el tiempo, el presupuesto, la disponibilidad de personal, 
    el equipo de hardware y software.
    \item \textbf{Requisitos legales y éticos:} el proyecto esta limitado 
    por los requisitos legales y éticos que deben cumplirse, por ejemplo, en lo 
    que respecta a la privacidad de los datos, la seguridad, la propiedad intelectual 
    y la ética en la investigación.
    \item \textbf{Limitación de acceso a datos:} el proyecto esta limitado
    por el acceso a los datos de los clientes de Tecnalia. En este caso, se
    utilizarán datos generados sinteticamente para poder realizar el proyecto.
\end{itemize}
\subsubsection{Reglas de Negocio}
\subsubsection{Alcance del Trabajo}
\subsubsection{Alcance del Producto}
\subsubsection{Catálogo de Requisitos Funcionales}
\subsubsection{Modelo del Dominio}
\subsubsection{Requisitos no Funcionales}
