\subsection{Plan de Pruebas}
Como parte del proceso de desarrollo del proyecto, se ha llevado a cabo un riguroso 
proceso de test para asegurar el correcto funcionamiento del mismo. Es importante 
destacar que, debido a la naturaleza de un proyecto de investigación, sólo se han 
testado las funcionalidades fundamentales del proyecto, es decir, aquellas que 
son críticas para el cumplimiento de los objetivos de la investigación. Es poco 
probable que estas funcionalidades sufran cambios significativos, por lo que se ha 
prestado especial atención en asegurar que estén libres de errores y funcionen correctamente.\medskip

Se ha utilizado Pytest\cite{pytest} que es un framework de testing de software para escribir, 
organizar y ejecutar pruebas unitarias automatizadas. Es una herramienta popular 
debido a su sintaxis sencilla y fácil de aprender, además de su capacidad para detectar 
reportar errores de manera clara y concisa. Pytest permite a los desarrolladores 
escribir pruebas más legibles, organizadas y escalable con menos código 
que otras alternativas. 

\begin{figure}[ht]
    \label{fig:pytest-output}
    \begin{lstlisting}
=========================== test session starts ===================
platform linux -- Python 3.10.6, pytest-7.3.0, pluggy-1.0.0
rootdir: /home/betagmr/dev/fjsp
configfile: pyproject.toml
collected 15 items                                                         

test/graph_test.py ...........                               [ 58%]
test/environment/fjspenv_test.py ......                      [ 89%]
test/model/eval_model_test.py .                              [ 94%]
test/virtualenv/version_test.py .                            [100%]

============================ 19 passed in 1.77s =================== 
    \end{lstlisting}
    \caption{Ejecución de las pruebas unitarias con Pytest}
\end{figure}

En la figura \ref{fig:pytest-output} se puede ver la salida de la ejecución de las
pruebas unitarias. En este caso, se han ejecutado 19 pruebas unitarias, de las cuales
ofrecen un 60\% de cobertura del código total y un 100\% de cobertura de las funcionalidades
fundamentales del proyecto. Se consideran funcionalidades fundamentales aquellas que
están relacionadas con la gestión de los grafos, el environment y la evaluación de los
modelos.
