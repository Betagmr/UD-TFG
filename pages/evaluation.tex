\section{Sistema de evaluación del modelo}
    Un buen sistema de evaluación es esencial para un proyecto de machine learning. 
    No solo es útil por el hecho de medir la precisión y el rendimiento de los modelos, 
    sino que además sirve como un identificador claro de nuestro progeso dentro del desarrollo. 
    El seleccionar el modelo óptimo en fases tempranas del proceso no es tan importante 
    como el saber que las iteraciones en las que gastamos tiempo y recursos estan teniendo un impacto 
    positivo en su aprendizaje. Aunque físicamente es imposible conseguir una productividad 
    perfecta, con ayuda de una metodología sólida es posible reducir al maximo esta problemática.
    
    \subsection{Selección de métrica}
    El objetivo de una métrica es, mediante ponderacion numérica, identificar errores en el modelo y 
    servir como ayuda para la toma de decisiones, cambios en el diseñe o la soluación de problemas. En el caso del 
    FJSP, el objetivo es encontrar una acorde a la tarea que queramos abordar. En la literatura se atacan dos puntos
    principalmente: \textit{minimizar el tiempo de finalización} y \textit{minimizar el tiempo de espera entre máquinas}. 

    \begin{itemize}
        \item \textbf{Minimizar el tiempo de espera entre máquinas:} mide el tiempo promedio que cada operación debe 
        esperar antes de ser procesado en una máquina específica. Esta métrica es importante porque 
        indica la eficiencia del sistema en términos de cómo se están utilizando los recursos disponibles 
        y cómo se gestionan los trabajos en las colas.
        \item \textbf{Minimizar el tiempo de finalización:} se basa en el tiempo total que se tarda en completar todos 
        los trabajos del sistema. Es un buen indicador porque evalúa el rendimiento global del sistema 
        y la eficacia para completar el proceso en el plazo establecido. 
    \end{itemize}

    Aunque ambas métricas son importantes, la segunda es la que se vamos a utilizar como referencia en cuanto al
    desempeño del modelo, ya que uno de nuestros objetivos principales es reducir los tiempos de producción y no tanto
    aprobechar las máquinas lo máximo posible. Es posible utilizar ambas para identificar puntos débiles tanto del modelo 
    como del propio proceso que se quiere optimizar pero, por la propia idiosincrasia de las mismas, el mejorar la puntuación
    en una de ellas inevitablemente afectara negativamente la puntuación de la otra en un amplio número de casuísticas.

    \subsection{Implementación de la métrica} 


\pagebreak