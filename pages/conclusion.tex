\section{Conclusiones y trabajo a futuro}
\subsection{Conclusiones}
El desarrollo completo de este proyecto me ha brindado un gran conocimiento 
y experiencia, representando una oportunidad invaluable para aprender y mejorar 
mis habilidades dentro del sector de la IA. Es el proyecto más grande en el que 
he trabajado hasta la fecha, lo que ha supuesto un gran desafío para llevarlo a cabo. 

\subsection{Trabajo a futuro}
A pesar de los avances significativos logrados mediante el enfoque de 
IL en combinación con redes neuronales de grafos para resolver el FJSP, 
existen varias oportunidades de mejora y direcciones prometedoras para 
investigaciones futuras. A continuación, se exploran algunas de estas 
áreas y se proponen posibles extensiones del trabajo realizado hasta ahora.

\begin{itemize}
    \item \textbf{Mezcla de Imitation Learning y Offline Learning:} En 
    el contexto del FJSP, el Imitation Learning ha demostrado ser una 
    técnica efectiva para aprender políticas de programación de trabajos 
    flexibles a partir de ejemplos expertos. Sin embargo, una posible mejora 
    sería combinar el Imitation Learning con técnicas de Offline Learning, 
    donde se podría utilizar información adicional para mejorar aún más la 
    calidad del entrenamiento. Gracias al offline learning sería posible no 
    solo aprender cual es la acción óptima a tomar, sino también aprender 
    de las acciones que no son tan buenas durante el periodo de entrenamiento, 
    lo permitiría capturar patrones más complejos y adaptarse mejor a las 
    diferentes configuraciones del problema.
   \item \textbf{Exploración mediante RL:} Otra dirección interesante sería explorar 
   la posibilidad de aplicar RL al modelo pre-entrenado obtenido mediante 
   el IL. Después de entrenar el modelo inicial con ejemplos expertos, 
   se podría realizar un proceso de fine-tuning utilizando RL para refinar 
   y ajustar la política de asignación de trabajos. Esto permitiría adaptar 
   el modelo a diferentes contextos y considerar aspectos dinámicos del 
   FJSP en tiempo real. Para ello, se podría utilizar un algoritmo de
   RL como PPO.
   \item \textbf{Extensión del enfoque a otras variantes del FJSP:} El 
   enfoque propuesto se ha centrado específicamente en el FJSP estándar. 
   Sin embargo, existen otras variantes del problema, como el Green Flexible 
   Job Shop, que incorporan restricciones adicionales relacionadas con la 
   eficiencia energética y la sostenibilidad. Sería interesante investigar 
   cómo se puede adaptar el enfoque basado en redes neuronales de grafos e 
   Imitation Learning para abordar estas variantes más complejas del FJSP y 
   considerar múltiples objetivos simultáneamente, como la minimización del 
   make span y el consumo energético.
\end{itemize}