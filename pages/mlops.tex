\section{Desarrollo basado en MLOps}
    MLOps (Machine Learning Operations) es una extensión de la metodología DevOps (Development Operations)
    que se enfoca en la integración, desarrollo y gestión del software de la mano del ciclo de vida del modelo.
    Uno de sus pilares es la automatización y en la mejora progresiva de la calidad del modelo, 
    lo que permite una implementación y puesta en producción mucho más rápida y eficiente. Para lograr esto,

    \subsection{Principios de MLOps}
    A continuación vamos a detallar los principios de MLOps que vamos a seguir en el desarrollo de nuestro proyecto.
    En esta sección se incluyen los puntos más importantes de la metodología, pero no vamos a profundiczar en ellos
    ya que se tratarán en profundidad en la sección correspondiente.
    
    \begin{itemize}
        \item \textbf{Automatización}: la automatización es la clave para la eficiencia y la escalabilidad.
        Aqui incluimos tareas como la generación de datos, el despliegue del modelo, la evaluación y 
        la puesta en producción. 
        \item \textbf{Versionado}: es importante tener un control de versiones de los datos, el código y los modelos. 
        Puede variar el método dependiendo de la herramienta que se utilice, pero existen estándares como Git para la gestión
        de código y GitHub/GitLab para el almacenamiento de los repositorios.
        \item \textbf{Reproducibilidad}: es necesario poder reproducir los resultados de forma consistente, puede ser
        complicado en Machine Learning debido a la naturaleza aleatoria de los algoritmos. Igualmente aquí se trataran
        las herramientas y medidas necesarias para lograrlo.
        \item \textbf{Monitorización}: es importante tener un control de los modelos en producción.
    \end{itemize}
    
    

\pagebreak