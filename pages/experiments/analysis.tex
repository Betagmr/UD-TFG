\subsection{Análisis de los resultados}
En esta sección se analizarán los resultados obtenidos durante las diferentes
fases del proceso de desarrollo del proyecto. Se compararán los resultados
obtenidos con los resultados de la literatura existente, y se establecerán
conclusiones y posibles implicaciones para futuras investigaciones.

\begin{table}[ht]
    \caption{Tiempos de procesamiento por cada versión del modelo} 
    \centering 
    \begin{tabular}{cccccccccccc}  

    \toprule
    \multirow{2}{*}{\parbox[c]{.2\linewidth}{\centering Instance Name}} & 
    \multicolumn{2}{c}{Modelo v0} && 
    \multicolumn{2}{c}{Modelo v1} && 
    \multicolumn{2}{c}{Modelo v2} && 
    \multicolumn{2}{c}{Modelo v3} \\ 

    \cmidrule{2-3} \cmidrule{5-6} \cmidrule{8-9} \cmidrule{11-12}
     & {\centering TMP} & {GAP} && {TMP} & {GAP} && {TMP} & {GAP} && {TMP} & {GAP} \\

    \midrule
    MK01 & 154 & 2.850 && 77 & 0.923 && 49 & 0.225 && \textbf{43} & \textbf{0.075} \\
    MK02 & 121 & 3.661 && 31 & 0.191 && 37 & 0.423 && \textbf{35} & \textbf{0.346} \\
    MK03 & 935 & 3.583 && 377 & 0.850 && 253 & 0.240 && \textbf{219} & \textbf{0.073} \\  
    MK04 & 173 & 1.877 && 90 & 0.494 && 87 & 0.450 && \textbf{77} & \textbf{0.283} \\
    MK05 & 1647 & 8.578 && 365 & 1.122 && \textbf{190} & \textbf{0.104} && 195 & 0.133 \\
    MK06 & 546 & 8.578 && 121 & 1.122 && \textbf{100} & \textbf{0.754} && \textbf{100} & \textbf{0.754} \\
    MK07 & 576 & 3.143 && 318 & 1.288 && 179 & 0.287 && \textbf{177} & \textbf{0.273} \\
    MK08 & 1992 & 2.809 && 652 & 0.246 && 591 & 0.130 && \textbf{540} & \textbf{0.032} \\
    MK09 & 1935 & 5.302 && 481 & 0.567 && \textbf{355} & \textbf{0.156} && 368 & 0.198 \\
    MK10 & 1548 & 6.898 && 350 & 0.786 && 334 & 0.704 && \textbf{301} & \textbf{0.535} \\
    \hline
        Calc time    & 03:00    &&& 01:03    &&& 01:17    &&& 01:37    \\
        AVG GAP      & 4.282    &&& 0.709    &&& 0.347    &&& 0.269    \\
    \hline
    
    \end{tabular}
\end{table}

En el Cuadro 7.2 se muestran los resultados obtenidos por las
diferentes versiones del modelo, y el tiempo total de procesamiento
para completar todo el proceso de benchmark. A continuación vamos a
explicar el desempeño de cada una de las versiones del modelo.
\begin{itemize}
    \item \textbf{Modelo v0:} Esta versión del modelo es la primera que se
        desarrolló, y la que se utilizó para realizar las pruebas de
        concepto. Como se puede observar, los resultados obtenidos son
        muy malos, y el tiempo de procesamiento es muy elevado. Esto
        se debe a que el modelo no está bien formulado, y es una base
        para el desarrollo e iteración de las siguientes versiones. Esta versión
        utiliza la representación del cambio de máquina explicada en la
        sección \textit{Otras representaciones del problema}.
    \item \textbf{Modelo v1:} La primera version del modelo ya se desarrolló 
        con el objetivo de obtener resultados. Como se
        puede observar, los resultados obtenidos son mucho mejores, y el
        tiempo de procesamiento se ha reducido significativamente. Esto se 
        debe a que se han realizado mejoras internas en el modelo y se ha aplicado una
        normalización más sofisticada a los datos de entrada, lo que
        ha permitido obtener mejorar el promedio de los resultados. Sin 
        embargo, el modelo sigue teniendo problemas para resolver algunas
        instancias, y el tiempo de procesamiento sigue siendo elevado. Ya que
        para prácticamente la mitad de las instancias, el gap es mayor a uno,
        lo que implica que el modelo se encuentra lejos de la solución óptima.
    \item \textbf{Modelo v2:} Para esta version hubo un cambio muy significativo 
        en el environment. Hasta este punto la forma en la que el grafo representaba
        el estado del problema era la explicada en la ultima subsección de
        \textit{Otras representaciones del problema}. Sin embargo, se decidió suprimir 
        el cambio de máquina y cambiar completamente el paradigma de predicción. En vez
        de predecir a nivel de operación, se predice a nivel de la relación existente 
        entre máquinas y operaciones. Esto ha permitido reducir la complejidad del 
        espacio de decisiones, y proveer al modelo de una información más completa de
        los diferentes posibilidades que igual antes no se tenían en cuenta. En este
        caso, se ha conseguido reducir el gap promedio a 0.347, la mitad que en la
        versión anterior y ninguna de las instancias de benchmark tiene un gap mayor
        a uno. Sin embargo el tiempo de procesamiento se ha incrementado, ya que el
        environment realiza una serie nueva de cálculos para poder obtener la
        representación del estado del problema.
    \item \textbf{Modelo v3:}: Por último, en esta versión se ha
        realizado un proceso de optimización y refinamiento para
        mejorar la precisión y el tiempo de procesamiento del modelo. 
        Con la optimización de hiperparámetros se ha conseguido reducir
        el tiempo de procesamiento hasta un tercio del tiempo original, esto
        sumado a la tarea de profiling y el uso de la memoria caché para el
        cálculo de las características, ha permitido reducir el tiempo de procesamiento
        hasta un 20\% en tiempo de cálculo del environment. Aprovechando la mejora
        en el tiempo de procesamiento, se ha implementado una arquitectura de
        ensemble, que permite ejecutar diferentes versiones del modelo y combinar
        las ventajas de cada uno para obtener mejores resultados. En este caso, se
        ha seleccionado aquellos modelos que han tenido un mejor desempeño en el
        proceso de validación, específicamente uno para cada tipo de tamaño de
        instancia. Esto ha permitido reducir el gap promedio hasta 0.269, y
        mejorar la consistencia de los resultados, sin afectar demasiado al
        tiempo de procesamiento. El único problema que se ha encontrado es que
        a la hora de comparar el impacto que cada modelo tiene sobre el resultado
        nos encontramos con que el modelo que tiene una mejor puntuación en 
        instancias grandes es el que aporta un menor valor al resultado final.
\end{itemize}

Una vez se ha analizado el desempeño de cada una de las versiones del modelo,
se procede a realizar una comparación entre los resultados obtenidos por el
mejor modelo y los resultados obtenidos por las diferentes reglas de decisión
y algoritmos de optimización que se usan como referencia dentro del 
paper \cite{pbrandimarte}.

\begin{table}[ht]
    \centering
    \begin{tabular}[ht]{|l|ccccc|}
        \hline
                  & spt & lwkr & mwkr & tabu & Modelo v3\\
        \hline
        MK01      & 65      & 76    & 54    & 49   & \textbf{43}  \\
        MK02      & 44      & 48    & 41    & 41   & \textbf{35}  \\
        MK03      & 397     & 506   & 296   & 263  & \textbf{219} \\
        MK04      & 109     & 121   & 89    & 89   & \textbf{77}  \\
        MK05      & 231     & 291   & \textbf{188} & \textbf{188}  & 195 \\
        MK06      & 128     & 143   & 139   & 128  & \textbf{100} \\
        MK07      & 188     & 238   & 262   & 188  & \textbf{177} \\
        MK08      & 670     & 1042  & 558   & \textbf{523}  & 540 \\
        MK09      & 573     & 723   & 536   & 444  & \textbf{368} \\
        MK10      & 536     & 627   & 524   & 363  & \textbf{301} \\
        \hline
    \end{tabular}
    \caption{Resultados de las reglas de ordenación, búsqueda tabú y el modelo v3}
\end{table}

Para analizar el modelo en base a los resultados, 
es necesario entender el significado de las siglas asociadas a cada 
columna. A continuación, una breve descripción de cada una de ellas:

\begin{itemize}
    \item \textbf{spt:} Shortest Processing Time (Tiempo de procesamiento 
    más corto). Esta regla de ordenación asigna prioridad a las tareas con 
    menor tiempo de procesamiento. Busca minimizar el tiempo total de procesamiento.
    \item \textbf{lwkr:} Least Work Remaining (Menor trabajo restante). Esta 
    regla de ordenación asigna prioridad a las tareas con menor cantidad de trabajo 
    restante por realizar. Busca equilibrar la carga de trabajo.
    \item \textbf{mwkr:} Most Work Remaining (mayor trabajo restante). Esta regla 
    de ordenación asigna prioridad a las tareas con mayor cantidad de trabajo restante 
    por realizar. Busca maximizar la utilización de recursos.
    \item \textbf{tabu:} Tabu Search (Búsqueda Tabú). Este es un algoritmo de optimización 
    metaheurística que busca mejorar una solución iterativamente al explorar y moverse por 
    el espacio de búsqueda de soluciones. 
\end{itemize}

Analizando los resultados presentes en el (Cuadro 7.3), podemos hacer las siguientes observaciones. En la mayoría de 
las instancias, el Modelo v3 logra tiempos de procesamiento más bajos en comparación con 
los otros métodos. Como son los casos de MK01, MK02, MK03, MK04, MK06, MK07, Mk09 y Mk10.
Sin embargo, en algunas instancias como la Mk05 y la MK08, otros métodos también obtienen 
tiempos competitivos que superan ligeramente al Modelo V3. Además, en MK08, la búsqueda tabú 
consigue encontrar la mejor solución, mientras que en MK05, tanto la búsqueda tabú como la
regla de ordenación mwkr obtienen tiempos de procesamiento más bajos que el Modelo V3.
Estos resultados sugieren que puede haber áreas de mejora dentro del modelo. Hablaremos
de esto en la sección de conclusiones y trabajo futuro.