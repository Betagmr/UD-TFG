\subsection{Análisis de los resultados}
En esta sección se analizarán los resultados obtenidos durante las diferentes
fases del proceso de desarrollo del proyecto. Se compararán los resultados
obtenidos con los resultados de la literatura existente, y se establecerán
conclusiones y posibles implicaciones para futuras investigaciones.\medskip

\begin{table}[ht]
    \caption{Tiempos de procesamiento estimados para cada producto} 
    \centering 
    \begin{tabular}{cccccccccccc}  

    \toprule
    \multirow{2}{*}{\parbox[c]{.2\linewidth}{\centering Instance Name}} & 
    \multicolumn{2}{c}{Modelo v0} && 
    \multicolumn{2}{c}{Modelo v1} && 
    \multicolumn{2}{c}{Modelo v2} && 
    \multicolumn{2}{c}{Modelo v3} \\ 

    \cmidrule{2-3} \cmidrule{5-6} \cmidrule{8-9} \cmidrule{11-12}
     & {\centering TMP} & {GAP} && {TMP} & {GAP} && {TMP} & {GAP} && {TMP} & {GAP} \\

    \midrule
    MK01 & 154 & 2.850 && 77 & 0.923 && 49 & 0.225 && \textbf{43} & \textbf{0.075} \\
    MK02 & 121 & 3.661 && 31 & 0.191 && 37 & 0.423 && \textbf{35} & \textbf{0.346} \\
    MK03 & 935 & 3.583 && 377 & 0.850 && 253 & 0.240 && \textbf{219} & \textbf{0.073} \\  
    MK04 & 173 & 1.877 && 90 & 0.494 && 87 & 0.450 && \textbf{77} & \textbf{0.283} \\
    MK05 & 1647 & 8.578 && 365 & 1.122 && \textbf{190} & \textbf{0.104} && 195 & 0.133 \\
    MK06 & 546 & 8.578 && 121 & 1.122 && 100 & 0.754 && \textbf{100} & \textbf{0.754} \\
    MK07 & 576 & 3.143 && 318 & 1.288 && 179 & 0.287 && \textbf{177} & \textbf{0.273} \\
    MK08 & 1992 & 2.809 && 652 & 0.246 && 591 & 0.130 && \textbf{540} & \textbf{0.032} \\
    MK09 & 1935 & 5.302 && 481 & 0.567 && \textbf{355} & \textbf{0.156} && 368 & 0.198 \\
    MK10 & 1548 & 6.898 && 350 & 0.786 && 334 & 0.704 && \textbf{301} & \textbf{0.535} \\
    \hline
        Calc time    & 03:00    &&& 01:03    &&& 01:17    &&& 01:37    \\
        AVG GAP      & 4.282    &&& 0.709    &&& 0.347    &&& 0.269    \\
    \hline
    
    \end{tabular}
\end{table}

En la siguiente tabla se muestran los resultados obtenidos por las
diferentes versiones del modelo, y el tiempo total de procesamiento
para completar todo el proceso de benchmark. A continuación vamos a
explicar el desempeño de cada una de las versiones del modelo.
\begin{itemize}
    \item \textbf{Modelo v0:} Esta versión del modelo es la primera que se
        desarrolló, y la que se utilizó para realizar las pruebas de
        concepto. Como se puede observar, los resultados obtenidos son
        muy malos, y el tiempo de procesamiento es muy elevado. Esto
        se debe a que el modelo no está bien formulado, y es una base
        para el desarrollo e iteración de las siguientes versiones.
    \item \textbf{Modelo v1:} Esta versión del modelo es la primera que
        se desarrolló con el objetivo de obtener resultados. Como se
        puede observar, los resultados obtenidos son muy buenos, y el
        tiempo de procesamiento es muy bajo. Esto se debe a que se han
        realizado mejoras internas en el modelo y se ha aplicado una
        normalización más sofisticada a los datos de entrada, lo que
        ha permitido obtener mejores resultados.
    \item \textbf{Modelo v2:} Para esta version hubo un cambio en el
        environment, se cambio el paradigma de predicción y en vez
        de predecir a nivel de operaciones, se predice a nivel de
        la relación entre maquinas y operaciones. Esto ha permitido
        reducir la complejidad del espacio de decisiones y obtener
        mejores resultados.
    \item \textbf{Modelo v3:}: Por último, en esta versión se ha
        realizado un proceso de optimización de hiperparámetros para obtener
        las mejores configuraciones posibles para el modelo. Además,
        se ha implementado la arquitectura del ensemble, que permite
        ejecutar diferentes versiones del modelo y combinar las
        ventajas de cada uno para obtener mejores resultados.
\end{itemize}

\begin{table}[ht]
    \centering
    \begin{tabular}[ht]{|l|ccccc|}
        \hline
                  & spt & lwkr & mwkr & tabu & Modelo v3\\
        \hline
        MK01      & 65      & 76    & 54    & 49   & 43  \\
        MK02      & 44      & 48    & 41    & 41   & 35  \\
        MK03      & 397     & 506   & 296   & 263  & 219 \\
        MK04      & 109     & 121   & 89    & 89   & 77  \\
        MK05      & 231     & 291   & 188   & 188  & 195 \\
        MK06      & 128     & 143   & 139   & 128  & 100 \\
        MK07      & 188     & 238   & 262   & 188  & 177 \\
        MK08      & 670     & 1042  & 558   & 523  & 540 \\
        MK09      & 573     & 723   & 536   & 444  & 368 \\
        MK10      & 536     & 627   & 524   & 363  & 301 \\
        \hline
    \end{tabular}
    \caption{Resultados de las reglas de ordenación}
\end{table}

Para analizar cuál es el mejor modelo en base a los resultados, 
es necesario entender el significado de las siglas asociadas a cada 
columna. A continuación, una breve descripción de cada una de ellas:
\begin{itemize}
    \item \textbf{spt:} Shortest Processing Time (Tiempo de procesamiento 
    más corto). Esta regla de ordenación asigna prioridad a las tareas con 
    menor tiempo de procesamiento. Busca minimizar el tiempo total de procesamiento.
    \item \textbf{lwkr:} Least Work Remaining (Menor trabajo restante). Esta 
    regla de ordenación asigna prioridad a las tareas con menor cantidad de trabajo 
    restante por realizar. Busca equilibrar la carga de trabajo.
    \item \textbf{mwkr:} Most Work Remaining (mayor trabajo restante). Esta regla 
    de ordenación asigna prioridad a las tareas con mayor cantidad de trabajo restante 
    por realizar. Busca maximizar la utilización de recursos.
    \item \textbf{tabu:} Tabu Search (Búsqueda Tabú). Este es un algoritmo de optimización 
    metaheurística que busca mejorar una solución iterativamente al explorar y moverse por 
    el espacio de búsqueda de soluciones. 
\end{itemize}

Observando los resultados, podemos hacer las siguientes observaciones. En la mayoría de 
las instancias, el Modelo V3 logra tiempos de procesamiento más bajos en comparación con 
los otros métodos. Por ejemplo, en MK01, MK02, MK04, MK06 y MK07, el Modelo V3 tiene 
tiempos más bajos que el resto de los demás métodos.\medskip

Sin embargo, en algunas instancias, otros métodos también obtienen tiempos competitivos. 
Por ejemplo, en MK03, el modelo "mwkr" tiene un tiempo de procesamiento más bajo que el Modelo V3.
En instancias como MK05 y MK07, el Modelo V3 tiene tiempos de procesamiento más altos en comparación 
con los otros modelos, lo que sugiere que puede haber áreas de mejora en la optimización para estas 
instancias específicas.\medskip

En general, el Modelo V3 parece ofrecer un buen rendimiento en términos de minimización de tiempos 
de procesamiento en comparación con los otros modelos en varias instancias.\medskip