\section{Herramientas de desarrollo}
    La siguiente sección se enfoca en presentar las herramientas utilizadas para el desarrollo y  
    mantenimiento del codigo fuente del proyecto. Los recursos que se mencionan, si vien ayudan positivamente 
    al ecosistema de trabajo y a la calidad del mismo, es importante destacar que en este apartado no se incluyen 
    detalles sobre ninguna tecnologia directamente relacionada con la implementación del modelo de Machine Learning. 
    Este aspecto se tratará en un apartado posterior del informe.

    \subsection{Jupyter Notebook}

    \subsection{Pyenv y Virtualenv}

    \subsection{Git}

    \subsection{Pip-tools}

    \subsection{Pytest}
    Pytest es una de las mejores opciones para el testing de software y la preferente en proyectos de Python. 
    Es la libreria más popular dentro del Machine Learning debido a su facilidad de uso y  
    permite escribir teses unitarios de forma funcional. Cuenta con una amplia variedad 
    de herramientas para la gestion de pruebas, algunos de sus utilidades más relevantes son las fixtures, que
    optimiza la reutilización de recursos; el mock, que encapsula la funcionalidad de una funcion para poder
    testearla de forma aislada y el coverage, que muestra que zonas del codigo están cubiertas por la suit de teses. 

    \subsection{Pre-commit}

    \subsection{Makefile}

    \subsection{Flake8}

    \subsection{Black}
    
    \subsection{Docker}
    

\pagebreak